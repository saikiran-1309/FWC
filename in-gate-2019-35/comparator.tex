\documentclass{article}
\usepackage{graphicx} % Required for inserting images
\usepackage{karnaugh-map}

\title{ASSIGNMENT}
\date{JULY 2023}
\author{Sai Kiran \\saikiran30407@gmail.com\\FWC22146\\IIT Hyderabad-Future Wireless Communication }


\date{}

\begin{document}


\maketitle
 \tableofcontents

\pagebreak
\section{Problem}
 Q.35.  If $X = X_1X_0$ and $Y = Y_1Y_0$ are 2-bit binary numbers. The Boolean function S that satisfies the condition "If $X \textgreater Y$, then $S= 1$", in its minimized form, is\\
 \begin{enumerate}
 
     \item[A.] $X_1Y_1$+$X_0Y_0$
     \item[B.] $X_1\overline{Y_1}+X_0\overline{Y_0}\overline{Y_1}+X_0\overline{Y_0}X_1$
     \item[C.] $X_1\overline{Y_1}x_0\overline{Y_0}$
     \item[D.] $X_1Y_1+X_0\overline{Y_0}Y_1+X_0\overline{Y_0}\overline{X_1}$
 \end{enumerate}
 
 \section{Components}
  \begin{table}[h]
  \centering
   \begin{tabular}{|c|c|c|}
   \hline
   \textbf{Component}& \textbf{Values} & \textbf{Quantity}\\
\hline
ArduinoUNO &  & 1 \\  
\hline
JumperWires& M-M & 6 \\ 
\hline
Breadboard &  & 1 \\
\hline
LED & &5 \\
\hline
Resistor &220ohms & 5\\
\hline
   \end{tabular}
   \end{table}


\section{Reduction of logical circuit}


    $X=X_1X_0$ $Y=Y_1Y_0$\\
    The boolean function $S$ that satisfies the condition,
    if X \textgreater Y, then  $S=1$.\\
    We can represent the above condition through a truth table.



\begin{table}[h]
\centering
\begin{tabular}{|c|c|c|}
\hline
$X$&$Y$&$S$\\$X_1X_0$ & $Y_1Y_0$ & $X\textgreater Y$\\
\hline
0   0 & 0  0 & 0 \\
\hline
0   0 & 0  1 & 0 \\
\hline
0   0 & 1  0 & 0 \\
\hline
0   0 & 1  1 & 0 \\
\hline
0   1 & 0  0 & 1 \\
\hline
0   1 & 0  1 & 0 \\
\hline
0   1 & 1  0 & 0 \\
\hline
0   1 & 1  1 & 0 \\
\hline
1   0 & 0  0 & 1 \\
\hline
1   0 & 0  1 & 1 \\
\hline
1   0 & 1  0 & 0 \\
\hline
1   0 & 1  1 & 0 \\
\hline
1   1 & 0  0 & 1 \\
\hline
1   1 & 0  1 & 1 \\
\hline
1   1 & 1  0 & 1 \\
\hline
1   1 & 1  1 & 0 \\
\hline
\end{tabular}
\caption{TRUTH TABLE FOR $S=X$\textgreater Y }
\end{table}

\pagebreak
By using k-maps, we can find the bolean expression for the condition, $S=1$ if, $X \textgreater Y$


\begin{karnaugh-map}[4][4][1][$Y_1Y_0$][$X_1X_0$]
\minterms{4,8,9,12,13,14}
\maxterms{0,1,2,3,5,6,7,10,11,15}

\implicant{4}{12}

\implicantedge{12}{12}{14}{14}

\implicantedge{8}{9}{12}{13}


\end{karnaugh-map}\\
Minimized form:
 $S=X_1$\(\overline{Y_1}\)+ $X_1X_0$\(\overline{Y_0}\)+$X_0$\(\overline{Y_1}\) \(\overline{Y_0}\) 

 \section{Implementation}
 \begin{table}[h]
     \centering
     \begin{tabular}{|c|c|c|}
        \hline arduino  &  & output  \\
         \hline 2 & resistor  &led 1 ($X_1$)\\
         \hline 3 & resistor  &led 2 ($X_0$)\\
         \hline 4 & resistor  &led 3 ($Y_1$)\\
         \hline 5 & resistor  &led 4 ($Y_0$)\\
         \hline 8 & resistor  &led 5 ($S$)\\
         \hline
     \end{tabular}
     \caption{}
     \label{tab:my_label}
 \end{table}
 \pagebreak

 \section{Procedure}
 \begin{enumerate}
     \item Connect the circuit as per the above table.
     \item The leds 1,2,3,4 represent the values of $X_1$,$X_0$,$Y_1$ and $Y_0$ respectively.
     \item The led 5 represent $S$. if the condition $X$ \textgreater $Y$ is true.
     \item Execute the circuits using the below code.
 \end{enumerate}
 \begin{table}[h]
	   \centering
	   \begin{tabular}{|c|}
	   \hline
	   https://github.com/saikiran-1309/FWC/blob/main/in-gate-2019-35/code/src/comp.cpp\\
	   \hline
	   \end{tabular}
   \end{table}


\end{document}